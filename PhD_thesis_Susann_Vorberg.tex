\documentclass[12pt,a4paper,twoside]{book}
\usepackage[utf8]{inputenc}
\usepackage{amsmath}
\usepackage{amsfonts}
\usepackage{amssymb}
\usepackage{graphicx}
\usepackage{siunitx}
\usepackage{acronym}
\usepackage[bookmarks=true,colorlinks=true,urlcolor=blue,citecolor=blue,linkcolor=blue,unicode=true]{hyperref}
\usepackage[top=2.5cm, bottom=2.5cm]{geometry}

\setlength{\parindent}{0pt}
\setlength{\parskip}{6pt plus 2pt minus 1pt}
\setlength{\emergencystretch}{3em}  % prevent overfull lines
\providecommand{\tightlist}{%
  \setlength{\itemsep}{0pt}\setlength{\parskip}{0pt}}


% Numbering of sections unto depth=5
\setcounter{secnumdepth}{5}

% Table of contents formatting
\renewcommand{\contentsname}{Table of Contents}
\setcounter{tocdepth}{1}

% Headers and page numbering 
\usepackage{fancyhdr}
\pagestyle{plain}

% Chapter styling
\usepackage[grey]{quotchap}
\makeatletter 
\renewcommand*{\chapnumfont}{%
  \usefont{T1}{\@defaultcnfont}{b}{n}\fontsize{80}{100}\selectfont% Default: 100/130
  \color{chaptergrey}%
}
\makeatother


%------------------- Definition of LMU title pages
\newcommand{\LMUCover}[3]{
    \thispagestyle{empty}
    {\parindent0cm \rule{\linewidth}{.7ex}}
    
    \begin{flushright}
      \vspace*{\stretch{1}}
      \sffamily\bfseries\Huge
      #1\\
      \vspace*{\stretch{1}}
      \sffamily\bfseries\large
      #2
      \vspace*{\stretch{1}}
    \end{flushright}
  
    \rule{\linewidth}{.7ex}
    \vspace*{\stretch{5}}
    \vspace*{\stretch{1}}
    
    \begin{center}\sffamily\LARGE{#3}\end{center}
}



\newcommand{\LMUTitlePage}[4]{
    \thispagestyle{empty}
    \vspace*{\stretch{1}}
    
    \begin{center}
      \Large Dissertation zur Erlangung des Doktorgrades der Fakultät für Chemie und Pharmazie der Ludwig-Maximilians-Universität München
    \end{center}
    
    \vspace*{\stretch{1}}
    {\parindent0cm \rule{\linewidth}{.7ex}}
    
    \begin{flushright}
      \vspace*{\stretch{1}}
      \sffamily\bfseries\Huge
      #1\\
      \vspace*{\stretch{1}}
    \end{flushright}
  
    \rule{\linewidth}{.7ex}

    \vspace*{\stretch{3}}
    \begin{center}
      \Large vorgelegt von\\
      \Large #2\\
      \Large geboren in #3\\
      \vspace*{\stretch{2}}
      \Large München, den #4
    \end{center}
}


\newcommand{\LMUErklaerung}[5]{
    \thispagestyle{empty}
    \begin{flushleft}
      \large \textbf{Erklärung} \\[1mm]
      \large Diese Dissertation wurde im Sinne von §7 der Promotionsordnung vom 28. November 2011 von #2 betreut.
      \bigskip
  
      \large \textbf{Eidesstattliche Versicherung}\\[1mm]
      \large Diese Dissertation wurde eigenständig und ohne unerlaubte Hilfe erarbeitet.
      \vspace{5em}
  
      \dots\dots\dots   \dots\dots\dots \hfill \dots\dots\dots\dots\dots\dots\dots\dots\\
      \large Ort, Datum \hfill #1
      \vfill
  
  
      \large Dissertation eingereicht am: \hfill #4
      \bigskip
    
      \large Erstgutachter:  #2 \hfill \dots\dots\dots\dots\dots\dots\dots
      \bigskip
    
      \large Zweitgutachter: #3 \hfill \dots\dots\dots\dots\dots\dots\dots
      \bigskip
    
      \large Tag der mündlichen Prüfung: \hfill #5
    \end{flushleft}
}


\usepackage{amsthm}
\newtheorem{theorem}{Theorem}[chapter]
\newtheorem{lemma}{Lemma}[chapter]
\theoremstyle{definition}
\newtheorem{definition}{Definition}[chapter]
\newtheorem{corollary}{Corollary}[chapter]
\newtheorem{proposition}{Proposition}[chapter]
\theoremstyle{definition}
\newtheorem{example}{Example}[chapter]
\theoremstyle{remark}
\newtheorem*{remark}{Remark}
\begin{document}

\frontmatter

%%% LMU cover page
\LMUCover
	{Bayesian Model for Prediction of Protein Residue-Residue Contacts}
	{Susann Vorberg}
	{15.10.2017}

\newpage
\thispagestyle{empty}
\cleardoublepage

%%% LMU title page
\LMUTitlePage
	{Bayesian Model for Prediction of Protein Residue-Residue Contacts}
	{Susann Vorberg}
	{Leipzig, Germany}
	{15.10.2017}

\newpage
\thispagestyle{empty}
\cleardoublepage

%%% LMU statement page
\LMUErklaerung
	{Susann Vorberg}
	{Dr. Johannes Soeding}
	{Prof. Dr. Julien Gagneur}
	{15.10.2017}
	{15.12.2017}

\newpage
\thispagestyle{empty}
\cleardoublepage
\frontmatter\setcounter{page}{1}

\chapter*{Summary}\label{summary}
\addcontentsline{toc}{chapter}{Summary}

Awesome contact prediction project abstract

\chapter*{Acknowledgements}\label{acknowledgements}
\addcontentsline{toc}{chapter}{Acknowledgements}

I thank the world.

\tableofcontents
\addcontentsline{toc}{chapter}{Table of Contents}

\mainmatter \setcounter{page}{1}

\chapter{Introduction}\label{introduction}

In his Nobel Prize speech in 1973
{[}\protect\hyperlink{ref-Anfinsen1973}{1}{]} Anfinsen postulated one of
the basic principles in molecular biology, which is known as
\emph{Anfinsen's dogma}: a protein's native structure is uniquely
determined by its amino acid sequence. With certain exceptions (e.g.
\(\ac{IDP}\) {[}\protect\hyperlink{ref-Wright1999}{2}{]}), this dogma
has proven to hold true for the majority of proteins.

Ever since, it is regarded as the biggest challenge in structural
bioinformatics {[}\protect\hyperlink{ref-Samish2015}{3}{]}, to realiably
predict a protein's structure given only its amino acid sequence.
\emph{De-novo} protein structure prediction methods use physical or
knowledge based energy potentials to find a protein conformation that
minimizes the protein's energy landscape. However, these methods are
limited by the complexity of the conformational space and the accuracy
of the energy potentials. Considering a protein with 150 amino acids,
that has approximately 450 degrees of freedom, Regarding the rotational
and translational degrees of freedom of the protein chain, the
complexity scales with XXX
{[}\protect\hyperlink{ref-Anfinsen1973}{1}{]}.

Far more successfull are methods that make use of existing structural
information. Since protein structure is conserved stronger than sequence
{[}\protect\hyperlink{ref-Anfinsen1973}{1}{]}, the structure of a
homologue protein can be used as template to model the target protein.
These template-based methods are crucially dependent on sensitive
sequence search tools that are able to detect even distantly related
proteins. Modern tools, like
HHblits{[}\protect\hyperlink{ref-Anfinsen1973}{1}{]}, are able to
identify remote homologs sharing as little as 10\% sequence identity
{[}\protect\hyperlink{ref-Anfinsen1973}{1}{]}. Of course, the
availability of a structural templates poses the main limitation for
this methodology.

Unfortunately, the number of solved protein structures increases only
slowly, as experimental methods are both time consuming and expensive
{[}\protect\hyperlink{ref-Dorn2014}{4}{]}. The
\(\ac{PDB}\){[}{\textbf{???}}{]} is the main repository for
marcomolecular structures and currently (Jul 2017) holds about 120 000
atomic models of proteins.

The primary technique for determining protein structures is X-ray
crystallography, accounting for roughly 90\% of entries in the
\(\ac{PDB}\). About 9\% of protein structures have been solved using
\(\ac{NMR}\) and less than 1\% using \(\ac{EM}\) (see FIG 1). All three
experimental techniques have advantages and limiations with respect to
certain modelling aspects. X-ray crystallography requires the protein to
form crystals, which is an arduous and time consuming process.
Furthermore, crystal packing forces the protein into a unnatural and
rigid environment preventing the observation of conformational
flexibility. \(\ac{NMR}\) studies the protein in an physiological
environment in solution and enables the study of protein dynamics as
ensembles of protein structures can be observed. However, \(\ac{NMR}\)
is limited to look at small proteins. Recently, \(\ac{EM}\) has
undergone a ``resolution revolution''
{[}\protect\hyperlink{ref-Egelman2016}{5}{]} and macromolecular
structures have been solved with resolutions up to 2A{[}citation{]}. The
limit of \(\ac{cryo-EM}\) lies in the size of proteins.

Compared to the tedious task of revealing atomic resolution of a protein
tertiary structure, it has become very easy to decipher the primary
sequence of proteins. With the latest sequencing technologies
{[}examples{]}, it takes only hours to sequence millions of basepaires
at low costs {[}example numbers{]} and the number of sequenced genomes
has risen tremendously. The UniProtKB {[}{\textbf{???}}{]}, the leading
resource for protein sequence information, contains more than 80 million
sequence entries (24 July 2017).

Consequently, the gap between the number of protein structures and
sequences is still growing and even new developments as single protein
structure determination {[}{\textbf{???}}{]} are not expected to close
this gap near in time. {[}Figure sequence structure gap{]}

Protein structure determines protein function. Therefore, structural
insights are of uttermost importance. They are essential for a detailed
understanding of chemical reactions, regulatory processes and transport
mechanisms. They are fundamental for the design of drugs and
antibiotics. Moreover structural abnormalities can lead to misfolding
and aggregation potentially causing diseases so studying them is
pathologically relevant.

The aformentioned trends illustrate the need of computational methods
and motivate research to solve \emph{Ansinsens Dogma} to reliably
predict protein structures from sequence alone.

\section{Structure Prediction}\label{structure-prediction}

Despite the knowledge of Anfinsen's postulate, we are not able to
reliably predict the structure of a protein from its sequence alone.
Generally it is assumed that a protein folds into a unique, well-defined
native structure that is near the global free energy minimum
(\autoref{fig:folding_funnel}). Levinthal's paradox
{[}\protect\hyperlink{ref-Levinthal1969}{6}{]} describes the complexity
of the folding process towards this minimum. It stresses the problem
that it is not possible for a protein to exhaustively search the
conformational space to get to its native fold. Due to the
``combinatorial explosion'' of possible conformations, an exhaustive
search would take unreasonably long. Hence, it is not a feasible
approach for structure prediction to scan all possible conformations.
Different approaches have been developed over time to overcome or elude
this problem.

\subsection{Template based methods}\label{template-based-methods}

Homology Modelling is applied in the case where homologous proteins with
solved structures are available and can be identified via sequence
alignments. The underlying assumption for this strategy relates to the
fact that structure is more conserved than sequence. That is, if we find
a related protein that shows sufficient similarity on the sequence
level, we can safely infer that both proteins share a similar structure.
Homology Modelling is assumed to yield reliable results when query and
target protein share more than 30\% sequence similarity, depending on
the sequence length (\textit{safe homology zone})
{[}{[}\protect\hyperlink{ref-Sander1991}{7}{]}\}. For proteins sharing
less than 30\% sequence similarity, the structure will not necessarily
be conserved. Within this \textit{twilight zone} of homology modelling
the number of false positives regarding structural similarity explodes
{[}{[}\protect\hyperlink{ref-Rost1999}{8}{]}\} and other structure
prediction methods have to be applied (see
\autoref{fig:schneider_sander_HM_threshold}). When a suitable target
structure has been identified, the backbone of the model is generated by
simply copying the coordinates of the target backbone atoms onto the
model. Non-aligned residues due to gaps in the alignment have to be
modelled \textit{de-novo}, meaning from scratch. This can be done by a
knowledge-based search for suitable fragments in the PDB or by true
energy-based \textit{de-novo} modelling. When the backbone is generated,
the side chains are modelled, usually by searching rotamer libraries for
energetically favoured residue conformations. Finally, the model is
energetically optimized in an iterative procedure. Force fields are
applied to correct the backbone and side chain conformations
{[}{[}\protect\hyperlink{ref-Gu2009}{9}{]}\}. By now, many automated
homology modelling servers are well-established (Modeller
{[}\textbf{???}\}, 3D-Jigsaw
{[}{[}\protect\hyperlink{ref-Bates2001}{10}{]}\}, SwissModel
{[}{[}\protect\hyperlink{ref-Arnold2006}{11}{]}\}) which allow more or
less manual intervention in the modelling process.

Homology modeling is by far the most successful approach to structure
prediction. It is applied in the case where homologous proteins with
solved structures are available and can be identified via sequence
alignments. The underlying assumption for this strategy relates to the
fact that structure is more conserved than sequence. That is, if we find
a related protein that shows sufficient similarity on the sequence
level, we can safely infer that both proteins share a similar structure.
Several automated homology modelling servers are well-established ,
e.g.~Modeller \citep{Eswar2007}, 3D-Jigsaw \citep{Bates2001} or
SwissModel \citep{Arnold2006}.

The limits of homology modeling lie in the identification of suitable
templates. As can be seen in \autoref{fig:pfam}, most protein families
have no known structure that can be used for homology modeling. In these
cases, other techniques, like fold recognition or ab initio predictions
might succeed.

Fold Recognition describes the inverse folding problem
\citep{Bowie1993}: instead of finding the compatible structure for a
given sequence, one tries to find sequences that fit onto a given
structure. Whether the query sequence fits a structure from the database
is not determined by sequence similarities but rather energetic or
environment specific measures. Thus, fold recognition methods are able
to recognize structural similarity even in the absence of sequence
similarity. The rationale basis for this strategy is the assumption that
the fold space is limited. It has been found that seemingly unrelated
proteins often adopt similar folds. This might be due to divergent
evolution (proteins are related, but homology cannot be detected at the
corresponding sequence level) or convergent evolution (functional
requirements lead to similar folds for unrelated proteins)
\citep{Gu2009}. Early approaches include profile based methods. Here,
the structural information of the protein is encoded into profiles,
which subsequently are aligned to the sequences
\citep{Bowie1991,Fischer1996,Ouzounis1993}. Advanced techniques are
known as ``threading'' techniques, describing the process of threading a
sequence through a structure and determining the optimal fit via energy
functions. \citep{Jones1992,Jones1998,Lemer1995}

Use a homologue protein structure as template. Only possible if a
homologue protein structure can be detected (usually via sequence
profile searches).

Threading techniques try to identify structural elements that fit to the
sequence.

\subsection{Template-free structure
prediction}\label{template-free-structure-prediction}

Ab initio or de-novo modeling techniques implement Anfinsen's Dogma most
closely in mimicking the folding process based only on physico-chemical
principles. Energy functions (physical or knowledge-based) are used to
describe the folding landscape and are minimized to arrive at the global
energy minimum corresponding to the native conformation. Since the
native conformation can be found near the global energy minimum of the
folding landscape, energy functions (physical or knowledge-based) have
been developed to describe this landscape. With respect to the idea of a
folding funnel, the energy function is minimized to mimic the folding
process that automatically leads to the global minimum. Again, there
exist numerous webservers that combine energy minimization, threading
techniques and fragment-based approaches, e.g.~Rosetta
\citep{Simons1999}, Tasser \citep{Zhang2004}, Touchstone II
\citep{Zhang2003}.

Drawbacks of these methods are the time requirements due to the
computational complexity of energy functions as well as their
inaccuracy.

Minimize a physical or knowledge-based energy function for the protein.
This has huge complexity due to large conformational space that needs to
be sampled.

\subsection{contact assisted denovo
predictions}\label{contact-assisted-denovo-predictions}

Structure Reconstruction from true contacts maps works well. Even a
small number of contacts is sufficient to reconstruct the fold of the
protein. Distance maps work even better.

What is the optimal distance cutoff to define a contact? Duarte et al
2010: between 8 and 12A Dyrka et al 2016 Konopka et al 2014 Sathyapriya
et al 2009

Many studies that successfuly predict structures denovo with the help of
predicted contact.

\section{Contact Prediction}\label{contact-prediction}

\subsection{Correlated mutations}\label{correlated-mutations}

contact prediction methods aim to identify correlated mutations from an
alignment of homologue protein sequences. main assumption is that two
interacting amino acid residues are coevolving: mutation of one of the
two residues can be compensated by mutation of the other residue

\subsection{Benchamrking methods}\label{benchamrking-methods}

\begin{itemize}
\item
  threshold for defining a contact: usually distance between \(C_\beta\)
  atoms (\(C_\alpha\) for Glycin) \textless{} 8 angstrom.
\item
  PPV: TP/(FP+TP) fraction of correct predictions among all predictions
\end{itemize}

\subsection{Pitfalls}\label{pitfalls}

Coevolution of residues can be mediated by molecules (e.g zinc ions) and
will not always imply spatial proximity in structure. Transitivity can
lead to correlation signals. Phylogenetic bias can also lead to
correlations. Sampling bias needs to be taken into account.

\section{State of the Art CP}\label{state-of-the-art-cp}

\subsection{Local methods}\label{local-methods}

MI and correlation measures suffer from transitivity of correlations

\subsection{Direct coupling analysis}\label{direct-coupling-analysis}

Can disentangle direct and indirect correlations.

Infer parameters of a maximum entropy model, more specifically a Potts
model (statistical physics) aka markov random fiels (computer science).
Likelihood function is convex, but Maximum Likelihood inference of model
is infeasible: Likelihood function needs to be reevaluated at each
iteration during optimization but partition function term sums over
20\^{}L sequences.

Many approximations: - mean field (Marks), Psicov - pseudo-likelihood -
belief-propagation (accurate but slow)

\subsection{Computing contact map from coupling
matrix}\label{computing-contact-map-from-coupling-matrix}

\begin{itemize}
\tightlist
\item
  direct information
\item
  frobenius norm
\end{itemize}

average product correction (also for MI)

(benchmark plot for localmethods + ccmpred)

\subsection{Meta-predictors}\label{meta-predictors}

\begin{itemize}
\tightlist
\item
  combining different approaches
\item
  jones et al: overlap between methods but also many unique predictions
\item
  machine learning methods incorporate sequence-derived features:
\item
  secondary structure predictions
\item
  solvent accessibilty
\item
  contact potentials
\item
  msa properties
\item
  pssms
\item
  physico-chemcial properties of amino acids
\end{itemize}

However, Meta-predictors will improve if basic methods improve.
Ultra-deep learning paper identifies coevolution features as crucial
feature.

\chapter{Methods}\label{methods}

all you need to know

\section{Dataset}\label{dataset}

\appendix


\chapter{Abbreviations}\label{abbreviations}

\begin{acronym}
    \acro{IDP}{intrinsically disordered proteins}
    \acro{APC}{Average Product Correction}
    \acro{CASP}{Critical Assessment of protein Structure Prediction}
    \acro{DCA}{Direct Coupling Analysis}
    \acro{DI}{Direct Information}
    \acro{EM}{electron microscopy}
    \acro{PDB}{protein data bank}
\end{acronym}

\backmatter

\listoffigures
\addcontentsline{toc}{chapter}{List of Figures}

\listoftables
\addcontentsline{toc}{chapter}{List of Tables}

\chapter*{References}\label{references}
\addcontentsline{toc}{chapter}{References}

\hypertarget{refs}{}
\hypertarget{ref-Anfinsen1973}{}
1. Anfinsen, C.B. (1973). Principles that Govern the Folding of Protein
Chains. Science (80-. ). \emph{181}, 223--230. Available at:
\url{http://www.sciencemag.org/content/181/4096/223}.

\hypertarget{ref-Wright1999}{}
2. Wright, P.E., and Dyson, H. (1999). Intrinsically unstructured
proteins: re-assessing the protein structure-function paradigm. J. Mol.
Biol. \emph{293}, 321--331. Available at:
\href{http://www.ncbi.nlm.nih.gov/pubmed/10550212\%20http://linkinghub.elsevier.com/retrieve/pii/S0022283699931108}{http://www.ncbi.nlm.nih.gov/pubmed/10550212 http://linkinghub.elsevier.com/retrieve/pii/S0022283699931108}.

\hypertarget{ref-Samish2015}{}
3. Samish, I., Bourne, P.E., and Najmanovich, R.J. (2015). Achievements
and challenges in structural bioinformatics and computational
biophysics. Bioinformatics \emph{31}, 146--150. Available at:
\href{https://oup.silverchair-cdn.com/oup/backfile/Content\%7B/_\%7Dpublic/Journal/bioinformatics/31/1/10.1093\%7B/_\%7Dbioinformatics\%7B/_\%7Dbtu769/2/btu769.pdf?Expires=1500051331\%7B/\&\%7DSignature=WYTe4F7OxZkkcnJmsjt-r8gIyHaHX5ANqUoy6w0PUuete2b\%7B~\%7D5ZU\%7B~\%7D\%7B~\%7DD6KOB1vq5A8MgCnrq3pDHUn0OSgz0QFmtU2RYf907-}{https://oup.silverchair-cdn.com/oup/backfile/Content\{\textbackslash{}\_\}public/Journal/bioinformatics/31/1/10.1093\{\textbackslash{}\_\}bioinformatics\{\textbackslash{}\_\}btu769/2/btu769.pdf?Expires=1500051331\{\textbackslash{}\&\}Signature=WYTe4F7OxZkkcnJmsjt-r8gIyHaHX5ANqUoy6w0PUuete2b\{\textasciitilde{}\}5ZU\{\textasciitilde{}\}\{\textasciitilde{}\}D6KOB1vq5A8MgCnrq3pDHUn0OSgz0QFmtU2RYf907-}.

\hypertarget{ref-Dorn2014}{}
4. Dorn, M., Silva, M.B. e, Buriol, L.S., and Lamb, L.C. (2014).
Three-dimensional protein structure prediction: Methods and
computational strategies. Comput. Biol. Chem. \emph{53}, 251--276.
Available at:
\url{http://linkinghub.elsevier.com/retrieve/pii/S1476927114001248}.

\hypertarget{ref-Egelman2016}{}
5. Egelman, E.H. (2016). The Current Revolution in Cryo-EM. Biophysj
\emph{110}, 1008--1012. Available at:
\url{http://www.cell.com/biophysj/pdf/S0006-3495(16)00142-9.pdf}.

\hypertarget{ref-Levinthal1969}{}
6. Levinthal, C. (1969). How to Fold Graciously. 22--24. Available at:
\url{http://www.citeulike.org/user/FBerkemeier/article/380320}.

\hypertarget{ref-Sander1991}{}
7. Sander, C., and Schneider, R. (1991). Database of homology-derived
protein structures and the structural meaning of sequence alignment.
Proteins \emph{9}, 56--68. Available at:
\url{http://www.ncbi.nlm.nih.gov/pubmed/2017436}.

\hypertarget{ref-Rost1999}{}
8. Rost, B. (1999). Twilight zone of protein sequence alignments.
Protein Eng. Des. Sel. \emph{12}, 85--94. Available at:
\url{http://peds.oxfordjournals.org/content/12/2/85.full}.

\hypertarget{ref-Gu2009}{}
9. Gu, J., and Bourne, P.E. (2009). Structural Bioinformatics
(Wiley-Blackwell) Available at:
\url{http://www.amazon.com/Structural-Bioinformatics-Jenny-Gu/dp/0470181052}.

\hypertarget{ref-Bates2001}{}
10. Bates, P.A., Kelley, L.A., MacCallum, R.M., and Sternberg, M.J.
(2001). Enhancement of protein modeling by human intervention in
applying the automatic programs 3D-JIGSAW and 3D-PSSM. Proteins
\emph{Suppl 5}, 39--46. Available at:
\url{http://www.ncbi.nlm.nih.gov/pubmed/11835480}.

\hypertarget{ref-Arnold2006}{}
11. Arnold, K., Bordoli, L., Kopp, J., and Schwede, T. (2006). The
SWISS-MODEL workspace: a web-based environment for protein structure
homology modelling. Bioinformatics \emph{22}, 195--201. Available at:
\url{http://bioinformatics.oxfordjournals.org/content/22/2/195.short}.


\end{document}
