\documentclass[12pt,a4paper,twoside]{book}
\usepackage[utf8]{inputenc}
\usepackage{amsmath}
\usepackage{amsfonts}
\usepackage{amssymb}
\usepackage{graphicx}
\usepackage{siunitx}
\usepackage{acronym}
\usepackage[bookmarks=true,colorlinks=true,urlcolor=blue,citecolor=blue,linkcolor=blue,unicode=true]{hyperref}
\usepackage[top=2.5cm, bottom=2.5cm]{geometry}

\setlength{\parindent}{0pt}
\setlength{\parskip}{6pt plus 2pt minus 1pt}
\setlength{\emergencystretch}{3em}  % prevent overfull lines
\providecommand{\tightlist}{%
  \setlength{\itemsep}{0pt}\setlength{\parskip}{0pt}}


% Numbering of sections unto depth=5
\setcounter{secnumdepth}{5}

% Table of contents formatting
\renewcommand{\contentsname}{Table of Contents}
\setcounter{tocdepth}{1}

% Headers and page numbering 
\usepackage{fancyhdr}
\pagestyle{plain}

% Chapter styling
\usepackage[grey]{quotchap}
\makeatletter 
\renewcommand*{\chapnumfont}{%
  \usefont{T1}{\@defaultcnfont}{b}{n}\fontsize{80}{100}\selectfont% Default: 100/130
  \color{chaptergrey}%
}
\makeatother


%------------------- Definition of LMU title pages
\newcommand{\LMUCover}[3]{
    \thispagestyle{empty}
    {\parindent0cm \rule{\linewidth}{.7ex}}
    
    \begin{flushright}
      \vspace*{\stretch{1}}
      \sffamily\bfseries\Huge
      #1\\
      \vspace*{\stretch{1}}
      \sffamily\bfseries\large
      #2
      \vspace*{\stretch{1}}
    \end{flushright}
  
    \rule{\linewidth}{.7ex}
    \vspace*{\stretch{5}}
    \vspace*{\stretch{1}}
    
    \begin{center}\sffamily\LARGE{#3}\end{center}
}



\newcommand{\LMUTitlePage}[4]{
    \thispagestyle{empty}
    \vspace*{\stretch{1}}
    
    \begin{center}
      \Large Dissertation zur Erlangung des Doktorgrades der Fakultät für Chemie und Pharmazie der Ludwig-Maximilians-Universität München
    \end{center}
    
    \vspace*{\stretch{1}}
    {\parindent0cm \rule{\linewidth}{.7ex}}
    
    \begin{flushright}
      \vspace*{\stretch{1}}
      \sffamily\bfseries\Huge
      #1\\
      \vspace*{\stretch{1}}
    \end{flushright}
  
    \rule{\linewidth}{.7ex}

    \vspace*{\stretch{3}}
    \begin{center}
      \Large vorgelegt von\\
      \Large #2\\
      \Large geboren in #3\\
      \vspace*{\stretch{2}}
      \Large München, den #4
    \end{center}
}


\newcommand{\LMUErklaerung}[5]{
    \thispagestyle{empty}
    \begin{flushleft}
      \large \textbf{Erklärung} \\[1mm]
      \large Diese Dissertation wurde im Sinne von §7 der Promotionsordnung vom 28. November 2011 von #2 betreut.
      \bigskip
  
      \large \textbf{Eidesstattliche Versicherung}\\[1mm]
      \large Diese Dissertation wurde eigenständig und ohne unerlaubte Hilfe erarbeitet.
      \vspace{5em}
  
      \dots\dots\dots   \dots\dots\dots \hfill \dots\dots\dots\dots\dots\dots\dots\dots\\
      \large Ort, Datum \hfill #1
      \vfill
  
  
      \large Dissertation eingereicht am: \hfill #4
      \bigskip
    
      \large Erstgutachter:  #2 \hfill \dots\dots\dots\dots\dots\dots\dots
      \bigskip
    
      \large Zweitgutachter: #3 \hfill \dots\dots\dots\dots\dots\dots\dots
      \bigskip
    
      \large Tag der mündlichen Prüfung: \hfill #5
    \end{flushleft}
}


\begin{document}

\frontmatter

%%% LMU cover page
\LMUCover
	{Bayesian Model for Prediction of Protein Residue-Residue Contacts}
	{Susann Vorberg}
	{15.10.2017}

\newpage
\thispagestyle{empty}
\cleardoublepage

%%% LMU title page
\LMUTitlePage
	{Bayesian Model for Prediction of Protein Residue-Residue Contacts}
	{Susann Vorberg}
	{Leipzig, Germany}
	{15.10.2017}

\newpage
\thispagestyle{empty}
\cleardoublepage

%%% LMU statement page
\LMUErklaerung
	{Susann Vorberg}
	{Dr. Johannes Soeding}
	{Prof. Dr. Julien Gagneur}
	{15.10.2017}
	{15.12.2017}

\newpage
\thispagestyle{empty}
\cleardoublepage
\frontmatter\setcounter{page}{1}

\chapter*{Summary}\label{summary}
\addcontentsline{toc}{chapter}{Summary}

Awesome contact prediction project abstract

\chapter*{Acknowledgements}\label{acknowledgements}
\addcontentsline{toc}{chapter}{Acknowledgements}

I thank the world.

\tableofcontents
\addcontentsline{toc}{chapter}{Table of Contents}

\mainmatter \setcounter{page}{1}

\chapter{Introduction}\label{introduction}

In his Nobel Prize speech in 1973
{[}\protect\hyperlink{ref-Anfinsen1973}{1}{]} Anfinsen postulated one of
the basic principles in molecular biology, which is known as
\emph{Anfinsen's dogma}: a protein's native structure is uniquely
determined by its amino acid sequence. With certain exceptions, this
dogma has proven to hold true for the majority of proteins.

\chapter{Methods}\label{methods}

all you need to know

\section{Dataset}\label{dataset}

\appendix


\chapter{Abbreviations}\label{abbreviations}

\begin{acronym}
    \acro{APC}{Average Product Correction}
    \acro{CASP}{Critical Assessment of protein Structure Prediction}
    \acro{DCA}{Direct Coupling Analysis}
    \acro{DI}{Direct Information}
    \acro{EM}{Expectation-Maximization}
\end{acronym}

\backmatter

\listoffigures
\addcontentsline{toc}{chapter}{List of Figures}

\listoftables
\addcontentsline{toc}{chapter}{List of Tables}

\chapter*{References}\label{references}
\addcontentsline{toc}{chapter}{References}

\hypertarget{refs}{}
\hypertarget{ref-Anfinsen1973}{}
1. Anfinsen, C.B. (1973). Principles that Govern the Folding of Protein
Chains. Science (80-. ). \emph{181}, 223--230. Available at:
\url{http://www.sciencemag.org/content/181/4096/223}.


\end{document}
